%\documentclass[a4paper,12pt,oneside,draft]{article}
\documentclass[a4paper,12pt,oneside]{article}

% In the original writelatex tamplate
\usepackage[english]{babel}
\usepackage[utf8]{inputenc}
\usepackage{amsmath}
\usepackage{graphicx}
\usepackage[colorinlistoftodos]{todonotes}

% By LoCigno
\usepackage{times}
\usepackage{graphicx}
\usepackage{subfigure}
\usepackage{csvsimple}
\usepackage{color}
\usepackage{url}
\usepackage{hyperref} 
\usepackage{cleveref}

% By Davide
\usepackage{comment}
\usepackage{booktabs}
\usepackage{color}

%Variables macros
\newcommand{\DefineVar}[2]{%
  \expandafter\newcommand\csname var-#1\endcsname{#2}%
} 
\newcommand{\var}[1]{\csname var-#1\endcsname}

\usepackage{courier}
\newcommand{\mono}[1]{\texttt{#1}}
%\newcommand{\mono}[1]{\texttt{\textbf{#1}}}

\title{Controller design for Lego Mindstorm motor}

\author{Diego Verona, Aliaksandr Siarohin, Mattia Digilio}

\date{\today}

\newtheorem{thm}[equation]{Theorem}

\begin{document}
%\maketitle
\makeatletter  % populates \@title, \@author, \@date
\begin{titlepage}
      \centering
      ~~~~~~~~~~~~~\\[-30mm]
      \includegraphics[keepaspectratio=true, width=7cm]{bg_eng_1r.jpg} \\[10mm]

     {
     \large \bfseries Master Degree in Computer Science\\[3mm] 
     Applied Robotics\\[3mm]
     AA 2015-2016
     }\\[10mm]

     %--------------------------------
     % Set the title, author, and date
     % 

     \vspace{0.5cm}
     {
     \Large \bfseries \textcolor{blue}{\@title} \par
     }
     \vspace{0.5cm}
%      {
%      \large {Group N. 1} \par
%      }
     \vspace{0.2cm}

     {\large {\@author}}
     \\ \vspace{.2cm}
     \@date

     \vspace{0.6cm}

    %-----------------------------------

\begin{abstract}

\textit{
  Report for the second assignment on Applied robotics: design and implement controller for the Lego NXT motor.\\In this report we show our controller, describe it properties and describe it digital implementation.
}


\end{abstract}

\end{titlepage}

\section{General definition}

\begin{thm}
\textbf{Root locus}. The \texttt{root locus}, or \texttt{Evans locus}, is a graphical method that
depicts the curves of the roots of the denominator of the
closed loop transfer function in the complex plane
(sometimes called Argand plane or Gauss plane). \\ The curves are parameterized by a parameter, typically the
gain of the loop  \footnote{\url{http://disi.unitn.it/~palopoli/courses/ECL/RootLocus.pdf}}
\end{thm}

\section{Design of continues time controller}
\subsection{Controller requirments}
The contoller should have:
\begin{itemize}
\item stady state tracking error = 0
\item overshot $<$ 20\%
\item settling time $<$ 0.4s
\end{itemize}
To show overshot requirment on root locus plot we use the folowing formula:
\begin{equation}
\frac{Re}{Im} = \frac{\xi}{\sqrt{1-\xi^{2}}} = \pm\frac{ln{0.2}}{\pi}
\end{equation}

To show settling time requirment, we use dominant pool aproximation:
\begin{equation}
Re = \frac{ln(\alpha)}{0.4}
\end{equation}

\subsection{Our design}
\begin{equation}
C(s) = \frac{(s+10)^2}{s(s+21)}
\end{equation}
\begin{equation}
K_c = 10
\end{equation}
You can see root locus in \cref{fig:root_locus}, and ideal responce to 1(t) \cref{fig:responce}. You can also see result of our scicoslab simulation  in \cref{fig:simulation}. Code is available in a shared folder\footnote{\url{https://github.com/AliaksandrSiarohin/AppliedRobotics/tree/master/controler}}.

\begin{figure}[t]
	\centering
	\includegraphics[width=\columnwidth]{../controler/root_locus}
	\caption{Root locus, red lines show constains on overshot and settling time}
	\label{fig:root_locus}
\end{figure}

\begin{figure}[t]
	\centering
	\includegraphics[width=\columnwidth]{../controler/response}
	\caption{Responce to 1(t).}
	\label{fig:responce}
\end{figure}

\begin{figure}[t]
	\centering
	\includegraphics[width=\columnwidth]{../motor_data/plots/filtering/90}
	\caption{Scicoslab simulation.}
	\label{fig:simulation}
\end{figure}

\section{Implimentation of digital controller}
Digital vestion of controller is (obtained using trapezoid rule):
\begin{multline}
y_{k+2} = \frac{1}{4 + 42 * T} (K_cu_{k+2}(4 + 100T^2 + 40T) + K_cu_{k+1}(-8 + 200T^2) \\+ K_cu_{k}(4 - 40T + 100T^2) + 8y_{k+1} - y_{k}(4 - 42*T))
\end{multline}
Speed estimated using exponential average
\begin{equation}
S(t) = 0.075 * S(t) + (1 - 0.075) * \frac{(Angle(t) - Angle(t-1))}{T}
\end{equation}

Code is available in a shared folder\footnote{\url{https://github.com/AliaksandrSiarohin/AppliedRobotics/tree/master/motor_controller}}.

\section{Conclusion}


\end{document}